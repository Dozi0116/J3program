\subsubsection{逐次実行型A/D変換器の原理}
図\ref{Block}に逐次変換型A/D変換器のブロック図を示す。
この方式では、A/D変換器の入力は、SAR(Successive Approximation Register)
と呼ばれるレジスタに接続されており、SARの示す値をD/A変換した電圧と、入力電圧とが
コンパレータで比較されることになる。ここで、コンパレータとは、+入力側と-入力側の
電圧を比較して、+入力側が大きければ出力がHighに-入力側が大きければLowを出力する目的で
作成された特殊なオペアンプの1つで、電圧比較器のことである。
このコンパレータの出力によってSARの各ビットはセット/リセットされ、
SARの示すビットがアナログ入力にもっとも近くなるようにコントロールされることにより
A/D変換を行っている。

%図を挿入

\subsubsection{逐次変換型A/D変換器の動作手順}
まず、A/D変換を行う条件として、入力電圧は変換中に変化しないものとし、初期状態として
SARの各ビットはすべてLowであるとする。最初のステップは、SARのMSBのみを
セット(High)して、この時の入力電圧とD/A変換器の出力電圧をコンパレータで比較する。
コンパレータ乃出力がHigh(入力電圧がD/A変換の出力電圧よりも大きい)の場合は、
そのままSARのビットをセットしたままとし、その逆であればリセットし、
そのビットの状態を決定する。
SARの対象ビットを1ビットずつLSB側へずらしながら同種の操作を繰り返す。
こうしてすべてのビットが定まれば、SARにA/D変換したディジタル値が得られる。その様子を
図\ref{Graph}に示す。この様に、MSBからLSBに向かって逐次的にSARのビットをセット/リセット
しながら順番にビットが定まるところから、このA/D変換方式のことを逐次変換型という。

%図を挿入
